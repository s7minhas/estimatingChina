\section*{Reviewer 2}

\subsection*{Major Comments}

\begin{enumerate}
	\item The paper is extremely well-done.  It uses state-of-the-art methods to tackle a timely and important research question. The findings have wide-ranging implications for the studies of authoritarian elites, winning coalitions, and selectorate theory.  The model offers a rare insight into the inner workings of a closed-off regime, and even provides some explanations for why individual actors may move in and out of favor with an authoritarian leader.  I think the paper is publishable with very minor revisions. In what follows, I offer some thoughts on framing, some clarification questions, and ideas for future directions.
	\begin{itemize}
		\item \textcolor{blue}{ \emph{
		We thank the reviewer for taking the time to provide their comments.
		}}
	\end{itemize}
	\item I would like to encourage the authors to discuss their data in more theoretical terms.  Why do elites make public appearances? How do they choose what events to appear at?
		\begin{itemize}
		\item \textcolor{blue}{ \emph{
			Insert great response.
			}}
	\end{itemize}

 \item Presumably, appearances at some events are expected. Are there protocols that call for attendance by government representatives? Are specific people expected to attend?  Even in the absence of information on the official protocols, the authors could show, for example, whether the same actors or actors in the same official positions attend the same events year after year?
	\begin{itemize}
		\item \textcolor{blue}{ \emph{
			Insert great response.
		}}
	\end{itemize}
	\item Appearances may also be a part of a broader strategy for career advancement, as they presumably provide visibility and networking opportunities. In this respect, it would be useful to know whether all events are open to all the elites or whether some are "by-invitation-only."  Again, in the absence of data, the authors could try to look at repeating events to see if there is variation in who attends.
	\begin{itemize}
		\item \textcolor{blue}{ \emph{
			Insert great response.
		}}
	\end{itemize}
	\item Are appearances necessary for promotion? Is attendance at some events indicative of a coming promotion? Is attendance at some events more important than others? Is there a way to account for some of this information in the statistical model or even in descriptive analysis? You could look for patterns in appearances preceding major promotions, even for a handful of select cases.
	\begin{itemize}
		\item \textcolor{blue}{ \emph{
			Insert great response.
		}}
	\end{itemize}
\end{enumerate}

\subsection*{Clarification Questions}

\begin{enumerate}
	\item The latent factor model places actors in a k-dimensional space, yet the visualization in Figure 5 is two-dimensional. D does this mean that k=2 in this application? Or did you only choose to show two dimensions?
	\begin{itemize}
		\item \textcolor{blue}{ \emph{
			Insert great response.
		}}
	\end{itemize}
	\item What is the range of the latent angle distance measure? In Figure 8, it looks like the range is between -1 and 1. Do higher values indicate greater dissimilarity of appearance patterns, yet not necessarily a tendency to appear at the same events?  It may be more intuitive to rescale so that higher values represent closeness.
	\begin{itemize}
		\item \textcolor{blue}{ \emph{
			Insert great response.
		}}
	\end{itemize}
	\item Future Research: Do you have data on known ties between actors, such as whether they went to the same school or any other variables that you could regress the latent angle distance on?  It would be fascinating it you could unpack some of the drivers of latent distance.
	\begin{itemize}
		\item \textcolor{blue}{ \emph{
			Insert great response.
		}}
	\end{itemize}
\end{enumerate}
