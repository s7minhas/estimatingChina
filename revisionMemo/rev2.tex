\section*{Reviewer 2}

\subsection*{Major Comments}

\begin{enumerate}
	\item The paper is extremely well-done.  It uses state-of-the-art methods to tackle a timely and important research question. The findings have wide-ranging implications for the studies of authoritarian elites, winning coalitions, and selectorate theory.  The model offers a rare insight into the inner workings of a closed-off regime, and even provides some explanations for why individual actors may move in and out of favor with an authoritarian leader.  I think the paper is publishable with very minor revisions. In what follows, I offer some thoughts on framing, some clarification questions, and ideas for future directions.
	\begin{itemize}
		\item \textcolor{blue}{ \emph{
		We thank the reviewer for taking the time to provide their comments.
		}}
	\end{itemize}
	\item I would like to encourage the authors to discuss their data in more theoretical terms.  Why do elites make public appearances? How do they choose what events to appear at?
		\begin{itemize}
		\item \textcolor{blue}{ \emph{
			We thank the reviewer for this comment and have provided additional discussion of our data. These changes have been made in the Data section of the manuscript (p. 10-15).  Specifically, we now note that public appearances can provide important signaling to both the general public and the regime insiders.  On the one hand, CCP elites are propelled to project an image of solidarity to the general public.  On the other hand, for the regime insiders, especially lower level officials peripheral to the power center, the public appearances of top elites serve to signal important policy and political changes. These signals, in turn, are key to factional struggles and policy implementations.   In other words, public appearances of CCP elites needs to not only be uniform enough to show party unity, but also sufficiently differentiating so as to reveal underlying elite affinities.
			}}
	\end{itemize}

 \item Presumably, appearances at some events are expected. Are there protocols that call for attendance by government representatives? Are specific people expected to attend?  Even in the absence of information on the official protocols, the authors could show, for example, whether the same actors or actors in the same official positions attend the same events year after year?
	\begin{itemize}
		\item \textcolor{blue}{ \emph{
			We thank the reviewer for pointing out this ambiguity.  Following the reviewer's suggestion, we have added a discussion of protocols related to the public appearances examined in this study on page 10.  Specifically, we highlight the distinction between two kinds of organizing protocols, that is, “by-rank-only” and “by-invitation-only.”  Most regular and ceremonial appearances like plenary sessions of the Central Committee (CC) follow the protocol of “by-rank-only.”  The attendance in these events are determined by formal party ranks (e.g., members of the CC Standing Committee, full CC members, and alternate CC members), and the extensive media coverage on these events help to present the unitary voice of the party.
		}}
	\end{itemize}
	\item Appearances may also be a part of a broader strategy for career advancement, as they presumably provide visibility and networking opportunities. In this respect, it would be useful to know whether all events are open to all the elites or whether some are "by-invitation-only."  Again, in the absence of data, the authors could try to look at repeating events to see if there is variation in who attends.
	\begin{itemize}
		\item \textcolor{blue}{ \emph{
			We thank the reviewer for the comment. Our study focuses on the events based on the protocol of “by-invitation-only,” which constitutes the large majority of public appearances of CCP elites but receives little public attention.  We did not include events based on the protocol of “by-rank-only” for two reasons.  First, our data has been collected in Xi’s first term, and there was no reshuffle in CC.  The attendance in these events, therefore, is largely uniform, conveying little information other than party ranks.  Second, despite their extensive media coverage, the number of these regular appearances is disproportionately small compared to events of “by-invitation-only.”  For instance, the CC plenary sessions runs on an annual basis, and there are only four occurrences in our dataset.
		}}
	\end{itemize}
	\item Are appearances necessary for promotion? Is attendance at some events indicative of a coming promotion? Is attendance at some events more important than others? Is there a way to account for some of this information in the statistical model or even in descriptive analysis? You could look for patterns in appearances preceding major promotions, even for a handful of select cases.
	\begin{itemize}
		\item \textcolor{blue}{ \emph{
			In order to help answer this question, we now highlight  a pair of cases to illustrate the importance of events to both promotion and policy on page 11 of the manuscript. Our example focuses on two policy domains, Xinjiang and Tibet.  Guo Shengkun, the police chief, and Wang Yang, the Vice Premier overseeing rural affairs and poverty reduction, have attended multiple meetings and events related to Xinjiang, neither was invited to any meetings related to Tibet.   On the other hand, we find Wang Yi, the Minister of Foreign Affairs, and Liu Qibao, the head of the CC Propaganda Department joined meetings related to Tibet but not Xinjiang.  These appearances help to revealing CCP’s striking policy differences in two seemingly similar domains.  While in Xinjiang CCP took an inward looking approach by focusing on public security and local economy, an outward looking approach is carried out in Tibet with the aim of winning international support.  Elites who were actively engaged in Xinjiang policies have gained important promotions.  In the reshuffle of the 19th CC in October 2017, Guo Shengkun was promoted to the full CC member, and Wang Yang to one of the seven members of the Politburo Standing Committee.
		}}
	\end{itemize}
\end{enumerate}

\subsection*{Clarification Questions}

\begin{enumerate}
	\item The latent factor model places actors in a k-dimensional space, yet the visualization in Figure 5 is two-dimensional. D does this mean that k=2 in this application? Or did you only choose to show two dimensions?
	\begin{itemize}
		\item \textcolor{blue}{ \emph{
			We thank the reviewer for pointing out the ambiguity here in terms of the dimensions of the latent space. The reviewer is indeed right that we used a 2 dimensional latent factor space. We have clarified this in the Latent Factor Analysis section of the manuscript.
		}}
	\end{itemize}
	\item What is the range of the latent angle distance measure? In Figure 8, it looks like the range is between -1 and 1. Do higher values indicate greater dissimilarity of appearance patterns, yet not necessarily a tendency to appear at the same events?  It may be more intuitive to rescale so that higher values represent closeness.
	\begin{itemize}
		\item \textcolor{blue}{ \emph{
			We thank the reviewer again for pointing out the ambiguity on our part, we have clarified the bounds and interpretation of our variable in the Latent Factor Analysis section of the manuscript. In terms of rescaling the measure, we understand the reviewer's reasoning for this but prefer to retain the current formulation where it can be interpreted as a measure of distance. However, if the reviewer or editor feels stronlgy about this then it is obviously an easy change that we would be happy to make.
		}}
	\end{itemize}
	\item Future Research: Do you have data on known ties between actors, such as whether they went to the same school or any other variables that you could regress the latent angle distance on?  It would be fascinating it you could unpack some of the drivers of latent distance.
	\begin{itemize}
		\item \textcolor{blue}{ \emph{
			We appreciate the reviewer's suggestions for future research.  We highlight these directions of future research in our revision.  Specifically, we suggested three lines of future studies.  First, in approximating the latent affinities among CCP elites, we treat their appearances between 2013 and 2017 as cross-sectional, which in turn could mask important temporal changes.  With more systematic data collection, future studies can benefit from longitudinal explorations of the evolution of the CCP elite network.  Second, we focus mainly on the latent relationship between the CCP elites.  This leaves certain questions for future research:  For example, is attendance at some events more important than others?  Is attendance at some events conducive to a future promotion?  Future studies can also benefit from adopting a bipartite framework, simultaneously looking at the characteristics and latent positions of both events and elites.
		}}
	\end{itemize}
\end{enumerate}
