\documentclass[11pt,english]{article}
\usepackage[utf8]{inputenc}
\usepackage{geometry}
\geometry{verbose,tmargin=1in,bmargin=1in,lmargin=1in,rmargin=1in}
\setcounter{secnumdepth}{2}
\setcounter{tocdepth}{-2}
\usepackage{float}
\usepackage{amsmath}
\usepackage{graphicx}
\usepackage{setspace}
\usepackage[authoryear]{natbib}

% Fonts

%\usepackage[default,osfigures,scale=0.95]{opensans}
\usepackage[T1]{fontenc}
\usepackage{ae}

% spacing
\usepackage{multirow}

\doublespacing

\makeatletter
%%%%%%%%%%%%%%%%%%%%%%%%%%%%%% User specified LaTeX commands.
\usepackage{dcolumn}
\usepackage{parskip}
\usepackage{booktabs}
\date{}
\usepackage[ragged, bottom, multiple]{footmisc}
\usepackage[colorlinks, urlcolor=blue, citecolor=blue, linkcolor=blue]{hyperref}

\@ifundefined{showcaptionsetup}{}{%
 \PassOptionsToPackage{caption=false}{subfig}}
\usepackage{subfig}
\makeatother

\usepackage{babel}
\begin{document}
\begin{spacing}{1}

\author{
  Huhe, Narisong\\
  \texttt{naris.huhe@strathclyde.ac.uk}
  \and
  Gallop, Max\\
  \texttt{max.gallop@strathclyde.ac.uk}
  \and
  Minhas, Shahryar\\
  \texttt{minhassh@msu.edu}
}

\title{\textbf{Who Are in Charge, Who Do I Work With, and Who Are My Friends:
A Latent Space Approach to Understanding Elite Coappearances in China}\thanks{The replication datasets and codes will be made available online. Corresponding author: Shahryar Minhas (minhassh@msu.edu).
}}


\maketitle
\begin{abstract}
\begin{flushleft}

How ruling elite arrange and maintain their power-sharing is key to our understanding of authoritarian politics. We analyze the dynamics of elite power-sharing in authoritarian regimes using a network framework that embeds actors onto a low-dimensional space. We also introduce a novel dataset tracking appearances of elite Chinese Community Party (CCP) members at political events. Our framework and data allow us to disentangle three key aspects of CCP elite power-sharing: (1) who are in charge, (2) who do I work with, and (3) who are my friends. Using a latent factor network analysis of approximately 10,000 appearance records of over 200 top CCP elites from 2013 to 2017, we empirically assess these three questions by computing elites' total appearances, dyadic coappearances, and their distance in a latent social space. We test how well these three indicators fare at predicting elites' appointments to the leading small groups (LSGs) of the CCP Central Committee and the Central Government, and from that analysis are able to highlight the need to account for the indirect ties elites share.

\end{flushleft}
\end{abstract}
\end{spacing}
[Word count: 9355]

\end{document}
